% Options for packages loaded elsewhere
\PassOptionsToPackage{unicode}{hyperref}
\PassOptionsToPackage{hyphens}{url}
\documentclass[
]{article}
\usepackage{xcolor}
\usepackage[margin=1in]{geometry}
\usepackage{amsmath,amssymb}
\setcounter{secnumdepth}{-\maxdimen} % remove section numbering
\usepackage{iftex}
\ifPDFTeX
  \usepackage[T1]{fontenc}
  \usepackage[utf8]{inputenc}
  \usepackage{textcomp} % provide euro and other symbols
\else % if luatex or xetex
  \usepackage{unicode-math} % this also loads fontspec
  \defaultfontfeatures{Scale=MatchLowercase}
  \defaultfontfeatures[\rmfamily]{Ligatures=TeX,Scale=1}
\fi
\usepackage{lmodern}
\ifPDFTeX\else
  % xetex/luatex font selection
\fi
% Use upquote if available, for straight quotes in verbatim environments
\IfFileExists{upquote.sty}{\usepackage{upquote}}{}
\IfFileExists{microtype.sty}{% use microtype if available
  \usepackage[]{microtype}
  \UseMicrotypeSet[protrusion]{basicmath} % disable protrusion for tt fonts
}{}
\makeatletter
\@ifundefined{KOMAClassName}{% if non-KOMA class
  \IfFileExists{parskip.sty}{%
    \usepackage{parskip}
  }{% else
    \setlength{\parindent}{0pt}
    \setlength{\parskip}{6pt plus 2pt minus 1pt}}
}{% if KOMA class
  \KOMAoptions{parskip=half}}
\makeatother
\usepackage{graphicx}
\makeatletter
\newsavebox\pandoc@box
\newcommand*\pandocbounded[1]{% scales image to fit in text height/width
  \sbox\pandoc@box{#1}%
  \Gscale@div\@tempa{\textheight}{\dimexpr\ht\pandoc@box+\dp\pandoc@box\relax}%
  \Gscale@div\@tempb{\linewidth}{\wd\pandoc@box}%
  \ifdim\@tempb\p@<\@tempa\p@\let\@tempa\@tempb\fi% select the smaller of both
  \ifdim\@tempa\p@<\p@\scalebox{\@tempa}{\usebox\pandoc@box}%
  \else\usebox{\pandoc@box}%
  \fi%
}
% Set default figure placement to htbp
\def\fps@figure{htbp}
\makeatother
\setlength{\emergencystretch}{3em} % prevent overfull lines
\providecommand{\tightlist}{%
  \setlength{\itemsep}{0pt}\setlength{\parskip}{0pt}}
\usepackage{bookmark}
\IfFileExists{xurl.sty}{\usepackage{xurl}}{} % add URL line breaks if available
\urlstyle{same}
\hypersetup{
  hidelinks,
  pdfcreator={LaTeX via pandoc}}

\author{}
\date{\vspace{-2.5em}}

\begin{document}

TODO: Add more to the title of your project here

rostools teaching R

This is a meaningless sentence This as well

\section{LearningR:Learning about reproduciblity in
R}\label{learningrlearning-about-reproduciblity-in-r}

TODO: Give a brief description of what your project is about

This project will learn me the basic of using R, using github and how to
produce reproducible code.

Forbedring

\section{Brief description of folder and file
contents}\label{brief-description-of-folder-and-file-contents}

TODO: As project evolves, add brief description of what is inside the
data, docs and R folders.

The following folders contain:

\begin{itemize}
\tightlist
\item
  \texttt{data/}:
\item
  \texttt{docs/}:
\item
  \texttt{R/}:
\end{itemize}

\section{Installing project R package
dependencies}\label{installing-project-r-package-dependencies}

If dependencies have been managed by using
\texttt{usethis::use\_package("packagename")} through the
\texttt{DESCRIPTION} file, installing dependencies is as easy as opening
the \texttt{LearningR.Rproj} file and running this command in the
console:

\begin{verbatim}
# install.packages("pak")
pak::pak()
\end{verbatim}

You'll need to have remotes installed for this to work.

\section{Resource}\label{resource}

dsa µµµµµ

For more information on this folder and file workflow and setup, check
out the \href{https://rostools.github.io/prodigenr}{prodigenr} online
documentation.

\end{document}
